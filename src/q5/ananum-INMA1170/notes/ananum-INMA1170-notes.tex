\documentclass[fr]{../../../eplnotes}

\newcommand{\Cn}{\C^n}
\newcommand{\Cnn}{\C^{n\times n}}
\newcommand{\V}{\mathcal{V}}


\usepackage{algorithm}
\usepackage{algorithmic}

\DeclareMathOperator{\sign}{sign}
\DeclareMathOperator{\rang}{rang}
\DeclareMathOperator{\myspan}{span}
\DeclareMathOperator{\diag}{diag}
\DeclareMathOperator{\col}{col}
\DeclareMathOperator{\myim}{Im}

\hypertitle{Analyse Numérique}{5}{INMA}{1170}
{Beno\^it Legat}
{Paul Van Dooren}

\paragraph{Remarque importante}
Ces notes ont été prises lors des cours magistraux
où Prof. Van Dooren estime que la partie correspondante du syllabus n'est pas complète.

Ce document est donc à prendre en complément au syllabus,
pas en remplacement du syllabus, ni en synthèse.

\section{Équations non-linéaires}
Étant donné $F: \Rn \to \Rn$ trouver $x \in \Rn$ tel que
\begin{align}
  \label{eq:Fx}
  F(x) = 0 \Rightarrow \begin{cases}
    F_1(x_1,\ldots,x_n) = 0\\
    F_2(x_1,\ldots,x_n) = 0\\
    \vdots\\
    F_n(x_1,\ldots,x_n) = 0
  \end{cases}
\end{align}

On va utiliser des méthodes itératives.
La fonction d'itération $f: \Rn\to\Rn$
\begin{align}
  \label{eq:xnfxn}
  x_{n+1} & = f(x_n) & k & = 0,1,2,\ldots
\end{align}

On voudrait que
$$\lim_{k\to\infty} x_n = s$$
résoud \eqref{eq:Fx}.
Ex: $f(x) = x - A^{-1}F(x)$ $\det(A) \neq 0$.

Point fixe $s = f(s)$ de \eqref{eq:xnfxn} résoud
\eqref{eq:Fx} puisque $F(s) = 0$.

\subsection{Condition de convergence}
On se restreint ici aux points d'attractions c'est à dire un point
$s$ tel qu'il existe un voisinage $V$ de $s$ tel que
$$\lim_{k\to\infty} x_k = s$$
pour tout $x_0 \in V$.

Le bassin d'attraction, c'est le plus grand ouvert d'attraction.

\begin{mydef}
  Le point fixe $s$ est
  \begin{enumerate}
    \item \emph{asymptotiquement} stable si il existe $\delta > 0$ tel que
      pour tout $\epsilon > 0$, on a
      \begin{align*}
        \|x_k - s\| & \leq \epsilon &
        \forall x_0 \in B_\delta(s) & \forall k \in \mathbb{N}.
      \end{align*}
    \item \emph{exponentiellement} stable s'il existe $C > 0$,
      $\alpha \in (0,1)$ tels que
      \begin{align*}
        \|x_k-s\| & \leq C\alpha^k\|x_0-s\| &
        \forall x_0 \in B_\delta(s) & \forall k \in \mathbb{N}.
      \end{align*}
    \item On parle de convergence \emph{globale} quand $V = \Rn$.
  \end{enumerate}
\end{mydef}

Comment obtenir ce type de convergence pour
\begin{equation}
  \label{eq:f}
  x_{k+1} = f(x_k).
\end{equation}

\begin{mydef}[Fonction contractante]
  $f:\Rn\to\Rn$ est contractante en $\Omega$
  si elle est Lipschitsienne avec un
  rapport $L < 1$
  \begin{align*}
    \|f(x) - f(y)\| \leq L\|x-y\| & \forall x,y \in \Omega.
  \end{align*}
\end{mydef}

\begin{mytheo}[Banach]
  Si $f$ est contractante dans $\Rn$ alors \ref{eq:f}
  admet un seul point fixe $s$ unique,
  avec convergence globale exponentiellement stable.
  \begin{proof}
    \begin{align*}
      \|x_k - s\| & = \|f(x_{k+1}) - f(s)\|\\
      & \leq L\|x_{k+1}-s\|\\
      & \leq \vdots\\
      & \leq L^k\|x_0-s\|.
    \end{align*}
  \end{proof}
\end{mytheo}

Pas souvent contractante dans $\Rn$.
Relaxons le théorème
\begin{mytheo}[Théorème relaxé]
  \label{theo:relax}
  Si $f$ est lipschitsienne dans $B_\epsilon(s)$,
  $\epsilon > 0$ et $0 \leq L < 1$ alors pour tout $x_0 \in B_\epsilon(s)$
  \begin{enumerate}
    \item les itérés restent dans $B_\epsilon(s)$;
    \item la suite converge exponentiellement vers $s$;
    \item $s$ est unique dans $B_\epsilon(s)$.
  \end{enumerate}
\end{mytheo}

\paragraph{Expansion de Taylor} et la Jacobienne.
Soit $f : \Rn \to \Rn$ alors si $D^{p+1}f$ est continu
$$f(x+k) = f(x) + Df(x) \cdot h + \cdots + \frac{D^pf(x)}{p!} + (h,\ldots,h)
+ R_p(h)$$
$R_p(h)$ est en $\bigoh\|h\|^{p+1}$.
$$Df(x) = J_f(x) =
\begin{pmatrix}
  \fpart{f_1(x)}{x_1} & \cdots & \fpart{f_1(x)}{x_n}\\
  \vdots & \ddots & \vdots\\
  \fpart{f_n(x)}{x_1} & \cdots & \fpart{f_n(x)}{x_n}
\end{pmatrix}.$$

$Df(x) \cdot h$ est linéaire en $h_1$.

\begin{mytheo}
  Si $f$ est dérivable dans $B_\epsilon(s)$ et $\|Df(x)\| = m < 1$
  $\forall x \in B_\epsilon$ alors \eqref{eq:f} satisfait
  le Théorème~\ref{theo:relax}.
\end{mytheo}

\begin{mytheo}
  Si $f$ est dérivable en $s \in \Rn$ et
  $$\|Df(s)\| = m < 1$$
  alors $s$ est attractif et exponentiellement stable.
\end{mytheo}

\begin{myrem}[Ostrowski]
  On peut remplacer $\|Df(s)\| = m < 1$ par $\rho(Df(s)) = \sigma < 1$.
\end{myrem}

\paragraph{Vitesse de convergence} (ordre de convergence)
Si $f:\Rn\to\Rn$ est $p$ fois continument dérivable et si
$Df(s) = D^2f(s) = \cdots = D^{p-1}f(s) = 0$ et $p \geq 2$
alors
$$\lim_{k\to\infty}\frac{\|x_{k+1}-s\|}{\|x_k-s\|^p} = c < \infty.$$
\begin{proof}
  $$f(x_k=s+h)
  = f(s) + \ldots + \frac{1}{p!}D^pf(s) \cdot (h,\ldots,h)+R_p(h)$$
  $x_{k+1} = f(x_k) = s + \bigoh\|x_k-s\|^p$
  $\|x_k+1\| - s = \bigoh(\|x_k-s\|^p)$.
\end{proof}

\begin{mydef}
  Pour des fonction d'itération avec expenstion de Taylor,
  \begin{itemize}
    \item ne permet que $p$ entier,
    \item la limite peut ne pas exister.
  \end{itemize}
\end{mydef}

\begin{mydef}[$p$ non entier]
  On remplace $\lim_{k\to\infty}$ par $\limsup_{k\to\infty}$.
  $$ \limsup y_k = \inf_{k} \sup_{l \geq k} y_l. $$

  L'ordre $p$ est défini comme ($p \geq 1$)
  $$ = \inf\left\{p \geq 1 \left|
  \limsup_{k\to\infty} \frac{\|x_{k+1}-s\|}{\|x_k-s\|}
  = \infty\right.\right\}$$

  plus petit $p$ tel que $\limsup$ n'est pas bornée.

  plus grand $p$ tel que $\limsup$ est bornée.
\end{mydef}
\begin{myexem}
  $|e_k| = 2^{-k^k}$ $k > 0$ $k = 1,2,\ldots$
  $$\limsup_{k\to\infty} \frac{|e_{k+1}|}{|e_k|^p}
  = \limsup_{k\to\infty} \frac{2^{-k^{k+1}}}{2^{-k^kp}}
  = \limsup_{k\to\infty} 2^{-k^k(k-p)} =
  \begin{cases}
    0 & \text{si } p < k\\
    1 & \text{si } p = k\\
    \infty & \text{si } p > k
  \end{cases}.$$
\end{myexem}

\begin{mydef}[Convergence superlinéaire]
  $$\limsup_{k\to\infty} \frac{\|x_{k+1}-s\|}{\|x_k-s\|^1} = 0$$
\end{mydef}

\begin{mydef}[Critère d'arrêt]
  On aimerait s'arrêter quand $\|F(x_k)\| \leq \epsilon_1$.
  Alternative $\|x_{k+1}-x_k\| \leq \epsilon_2$.
  Idéalement, on veut $\|x_{k+1}-s\| \leq \epsilon_3$.

  Supponsons que la convergence est rapide
  $$\|x_{k+1}-s\| \ll \|x_k-s\|.$$
  On a
  \begin{align*}
    x_{k+1} - s & = (x_{k+1} - x_k) + (x_k -s)\\
    \|x_{k+1} - x_k\| & \approx \|x_k - s\|.
  \end{align*}
  Quand on a des bornes sur $\|DF(s)=c_1\|$ et $\|DF(s)^{-1}=c_2\|$ alors, par Taylor,
  \begin{align*}
    F(x_k) & = DF(s)(x_k-s) + \bigoh(\|x_k-s\|^2)
  \end{align*}
  On néglige le $\bigoh(\|x_k-s\|^2)$ et par Cauchy-Schwartz
  \[ \|x_k-s\| \leq \|DF(s)^{-1}\|\cdot\|F(x_k)\|. \]
  On a aussi
  \[ x_k - s \approx (DF(s))^{-1}F(x_k) \]
  et par Cauchy-Schwartz
  \[ \|F(x_k)\| \leq \|DF(s)\|\cdot\|x_k-s\| \]
  d'où
  \[ \frac{\|F(x_k)\|}{c_1} \leq \|x_k-s\| \leq c_2\|F(x_k)\|. \]
\end{mydef}

\subsection{Méthodes du premier ordre}
$$x_{k+1} = x_k - A^{-1}F(x_k)$$
où $\det A \neq 0$.

Pour $n = 1$, $\alpha = A \neq 0$.
$f(x) = x - \alpha^{-1}F(x)$,
$|f'(s) < 1 \Rightarrow |1 - \alpha^{-1}F(s)| < 1$.
Il faut prendre $\sign \alpha = \sign F'(s)$, $0 < \alpha^{-1}F'(s) < 2$,
$\alpha > \frac{2}{F(s)}$.

Cette méthode cherche le zéro de
$$F(x) = F(x_k) + A(x-x_k).$$

\subsection{Méthode de Newton}
Chercher le zéro de
$$F(x) = F(x_k) + DF(x_k)(x-x_k)$$
On a la fonction d'itération
$$x_{k+1} = x_k - \frac{F(x_k)}{F'(x_k)}$$
pour $n = 1$.

On vérifie que la convergence est quadratique parce que $f'(s) = 0$.
\begin{align*}
  f(x) & = x - \frac{F(x)}{F'(x)}\\
  f'(s) & = 1 - \frac{F'(s)}{F'(s)} + \frac{F(s)F''(s)}{F'(s)^2} = 0
\end{align*}
pour autant que $F'(s) \neq 0$ (se rappeler que $F(s) = 0$).

\begin{mytheo}[Fourier]
  \begin{enumerate}
    \item $F \in C^2[a,b]$,
    \item $F(a)F(b) < 0$,
    \item $F'(x) \neq 0$ dans $[a,b]$,
    \item $F''(x) \geq 0,\leq 0$ dans $[a,b]$ (elle ne change pas de signe),
    \item $\left|\frac{F(a)}{F(b)}\right| \leq b-a$ $c = a$ ou $b$
      (à la première étape, je reste dans l'intervalle).
  \end{enumerate}
  Spdg, $F''(x) \geq 0$.
  \begin{itemize}
    \item Si $x_0 \in [s;b]$, alors $x_1 \in [a,s]$;
    \item Si $x_0 \in [a,s]$, alors $x_1 \in [x_0,s]$.
  \end{itemize}
  Convergence monotone à partir de $x_1$ vers $s$.
  (Preuve page 70 à savoir faire !)
\end{mytheo}

\paragraph{Exercice}
Si
$F(x) = (x-s)^p G(x)$
où $G(s) \neq 0$, alors $F'(s) = 0$.
Ceci montre que
$$x_{k+1} = x_k + p\frac{F(x_k)}{F'(x_k)}$$
converge quadratiquement.
\begin{proof}
  $f'(s) = 0$.
\end{proof}

% cours suivant
Halley
L'ordre de convergence est 3.
\begin{proof}
  Soit $f$ tel que $x_{k+1} = f(x_k)$, on montre que
  \begin{align*}
    f'(s) & = 0\\
    f''(s) & = 0.
  \end{align*}
\end{proof}

Why pas linéaire ?
$|L| < 1$
\begin{align*}
  e_{n+1} & = Le_n\\
  e'_n & = ce_n^2\\
  e'_{n+1} & = ce_{n+1}^2\\
  e'_{n+1} & = L^2e'_n
\end{align*}
$|L|^2 < 1$

\section{Introduction au chaos}
7 novembre 2013
On regardait l'itération
$$x_{k+1} = f(x_k)$$
avec $s = f(s)$.
\begin{align*}
  x_{k+1} & = f(s + h)\\
          & = f(s) + f'(s)h + \frac{f''(s)}{2}h^2\\
          & \approx s + f'(s)h.
\end{align*}

Point fixe attractif si $|f'(s)| < 1$.
Point fixe est
\begin{itemize}
  \item attractif si $|f'(s)| < 1$;
  \item répulsif si $|f'(s)| > 1$;
  \item neutre si $|f'(s)| = 1$.
\end{itemize}

\begin{myexem}
  $s = 0$

  \begin{center}
    \begin{tabular}{llll}
      $f(x)$ & $f'(x)$ & $f'(s)$ & \\
      $x + x^3$ & $1 + 3x^2$ & 1 & faiblement répulsif\\
      $x - x^3$ & $1 - 3x^2$ & 1 & faiblement attractif\\
      $x - x^2$ & $1 - 2x$ & 1 & faiblement répulsif/attractif
    \end{tabular}
  \end{center}
\end{myexem}
\begin{myexem}
  Équation logistique $f(x) = \mu x(1-x)$
  $0 \leq \mu \leq 4$.
  L'itération $x_{k+1} = f(x_k)$ $0 \leq x \leq 1$.
  Dessin dans notes de cours.

  Pour $0 \leq \mu \leq 4$, $f([0,1]) \subseteq [0,1]$.

  Points fixes: $x = \mu x(1-x)$
  $s_0 = 0$ et $s_\mu = 1 - 1/\mu$.

  \begin{itemize}
    \item Pour $\mu = 0$, $s_0 = 0$ (point fixe attractif).
    \item Pour $0 < \mu < 1$,
      \begin{tabular}{lll}
        $s_0 = 0$ & $f'(s) = \mu$ & attractif\\
        $s_\mu = 1 - 1/\mu < 0$.
      \end{tabular}
    \item Pour $\mu = 1$
      \begin{tabular}{lll}
        $s_0 = 0 = s_\mu$ & $f'(s_0) = 1$ & neutre faiblement attractif
      \end{tabular}
    \item $1 < \mu < 3$
      \begin{tabular}{lll}
        $s_0 = 0$ & $|f'(s_0)| = \mu$ & répulsif\\
        $s_\mu = 1 - 1/\mu$ & $|f'(s_\mu)| = |2-\mu| < 1$ & attractif
      \end{tabular}
    \item $\mu = 3$
      \begin{tabular}{ll}
        $s_0$ & reste répulsif\\
        $s_\mu$ & reste neutre
      \end{tabular}
    \item $3 < \mu < 4$, deux point fixes répulsifs.
  \end{itemize}
  \begin{align*}
    x_{k+2} & = f(f(x_k))\\
            & = \mu (\mu x(1-x))(1 - \mu x(1-x))\\
            & = \mu^2 x(1-x)(1 - \mu x(1-x))
  \end{align*}
  Les points fixes de ceci sont solutions de
  $$x = \mu^2 x(1-x)(1 - \mu x(1-x))$$
\end{myexem}

\begin{mydef}
  Si $V \subseteq \R$ $f(V) \subseteq V$ alors $f$ est chaotique si
  \begin{itemize}
    \item sensible aux conditions initiales
      \begin{align*}
        |x-y| & < \epsilon, & \exists n, \delta & > 0 & |f^n(x) - f^n(y)| & > \delta
      \end{align*}
    \item Topoliquement transitif
      \begin{align*}
        X,Y & \subseteq V & \exists n & : f^n(X) \cap Y \neq \emptyset
      \end{align*}
    \item Les points sont périodique sont denses dans $V$.
  \end{itemize}
\end{mydef}

\begin{myexem}
  \begin{align*}
    g(\theta) & = \left.2\theta\right|_{\pmod{2\pi}} & \theta \in [0,2\pi[\\
    g(S) & \subseteq S
  \end{align*}
  \begin{itemize}
    \item sensible OK
    \item topologiquement transitive OK car l'intervale double de taille à chaque fois donc à partir d'un certain
      $n$, il vaudra tout $[0,2\pi[$.
    \item $\exists k$ tel que $\theta = 2\theta - 2k\pi$, c'est $\theta = 2k\pi$ $k = 0$.
      Le seul point fixe de période 1 est $\theta = 0$.
      Pour ceux de période 2, il faut $\theta = 4\theta - 2k\pi$. Tout cela donne
      \begin{align*}
        \theta & = 2\theta - 2k\pi & \theta & = 2k\pi & \theta & = 0 & k & = 0\\
        \theta & = 4\theta - 2k\pi & 3\theta & = 2k\pi & \theta & = \frac{2\pi}{3},\frac{4\pi}{3}\\
        \theta & = 8\theta - 2k\pi & 7\theta & = 2k\pi & \theta & = \frac{2\pi}{7},\frac{4\pi}{7},\frac{6\pi}{7}
      \end{align*}
  \end{itemize}
\end{myexem}

\begin{myexem}
  $$
  \begin{array}{rcccl}
    & S & \stackrel{g}{\rightarrow} & S&\\
    \cos\theta& \downarrow & & \downarrow &\cos\theta\\
    %[-1,1] & \stackrel{q(t) = 2t^2-1}{\rightarrow} & [-1,1]\\
              & -1,1 & \stackrel{q(t) = 2t^2-1}{\rightarrow} & -1,1&\\
    \frac{1}{2}(1-t)& \downarrow & & \downarrow &\frac{1}{2}(1-t)\\
                    & 0,1 & \stackrel{f(x)=4x(1-x)}{\rightarrow} & 0,1&
  \end{array}
  $$
\end{myexem}

\section{Valeurs propres et vecteurs propres}
Pour $A \in \Cnn$ on dit que $\lambda \in \Lambda(A)$ si
\begin{align*}
  Ar & = \lambda v & v & \neq 0
\end{align*}
spécifie une pair $(\lambda,v)$ qui satisfait
\begin{align*}
  (A - \lambda I)v & = 0 & v & \in \ker(A - \lambda I)
\end{align*}
$\lambda$ satisfait $\det(A - \lambda I) = 0$.

$p(\lambda) = \det(\lambda I - A)$ est le polynôme caractéristique.

$\V = \ker(A - \lambda I)$ est un espace de dimension $n - k$ où $k$ est le $\rang(A - \lambda I)$.
$\V$ est l'``espace'' de vecteurs propres de $\lambda$.

\subsection{Espace invariant}
L'espace
$$A\V = \{Ax, x \in \V\}.$$

$\V$ est invariant sous $A$ si
$$A \V \subseteq \V.$$

\begin{myexem}
  $\myspan\{v_1,v_2\}$
  $Av_1 = \lambda_1v_1$
  $Av_2 = \lambda_2v_2$
  $A (\alpha v_1 + \beta v_2) = (\alpha\lambda_1 v_1 + \beta \lambda_2 v_2)$
\end{myexem}

\begin{align*}
  \V & = \ker(A - \lambda I)\\
  (A - \lambda I) \V & = 0
\end{align*}

$(A\V = \lambda \V \subseteq \V)$ (abus de notation).

\begin{mydef}
  $\lambda_0 \in \Lambda(A)$
  \begin{description}
    \item[multiplicité algébrique] multiplicité de $\lambda_0$ dans $p(\lambda)$.
    \item[multiplicité géométrique] $\dim\ker(A-\lambda_0 I)$
  \end{description}
\end{mydef}

\begin{myexem}
  $$
  A = \begin{pmatrix}
    0 & 1\\
      & 0 & 1\\
      &   & 0\\
      &   &   & 0 & 1\\
      &   &   &   & 0
  \end{pmatrix}
  $$
  $\lambda_0 = 0$ $p(\lambda) = \lambda^5$
  $$
  \V = \myspan(
  \begin{pmatrix}
    1\\0\\0\\0\\0
  \end{pmatrix},
  \begin{pmatrix}
    0\\0\\0\\1\\0
  \end{pmatrix}
  )
  $$
\end{myexem}

\begin{mytheo}[Jordan]
  Soit $A \in \Cnn$ $\Lambda(A) = \{\lambda_1,\ldots,\lambda_n\}$
  il existe $X$ tel que $\det(X) \neq 0$ tel que
  $$X^{-1}AX = J = \diag\{J_1,\ldots,J_k\}$$
  avec
  $$J_i = \diag\{J_{m_{i_1}}(\lambda_i),\ldots,J_{m{i_{l_i}}}(\lambda_i)\}$$
\end{mytheo}

\begin{myexem}
  \begin{itemize}
    \item multiplicité algébrique de $\lambda_i = \sum m_{ij}$
    \item multiplicité géométrique de $\lambda_i = l_i$
  \end{itemize}
\end{myexem}

\begin{myalgo}
  Pour $\lambda_1$ (dominante)
  \begin{algorithm}
    \caption{Méthode de la puissance $\|x_0\| = 1$ ($x_0$ donné)}
    \label{algo:puissance}
    \begin{algorithmic}
      \FOR{k = 0,1,\ldots}
      \STATE $z_{k+1} = Ax_k$
      \STATE $x_{k+1} = z_{k+1}/\|z_{k+1}\|$
      \ENDFOR
    \end{algorithmic}
  \end{algorithm}
  Ceci génère $x_0, Ax_0, A^2x_0, \ldots$ normalisé.

  Pour analyser al convergence, on regarde l'angle
  $$\angle(x,v) = \arccos\frac{|x^*v|}{\|v\|\|v\|} = \theta$$
  $0 \leq \theta < \pi$.
\end{myalgo}

\begin{mytheo}
  Supposons $A \in \Cnn$ diagonalisable ($A = T^{-1} \Lambda T$) et $|\lambda_1| > |\lambda_2|$
  alors la méthode de la puissance converge
  vers $v_1$ de façon linéaire our presque tout $x_0$.
  $$\lim_{k \to 0} \angle(x_k,v_1) = 0$$
  où $Av_1 = \lambda_1v_1$.
  \begin{proof}
    $$x_0 = \sum_{i=1}^n a_iv_i.$$
    car $A$ est diagonalisable donc les vecteurs propres engendrent $\Cn$.
    $$A^kx_0 = a_1\lambda_1^k\left[v_1 + \sum_{j=2}^n\frac{a_j}{a_i}\frac{\lambda_j^k}{\lambda_i^k}v_j\right]$$
    $\frac{\lambda_j}{\lambda_i}< 1$ donc $(\lambda_j/\lambda_i)^k$ tend vers 0 par rapport à $v_1$.
    $$A^kx_0 = a_1\lambda_1^k\left[v_1 + \bigoh(\left|\frac{\lambda_2}{\lambda_1}\right|)^k\right]$$
    $$\angle(v_1,x_k) = \bigoh\left(\left|\frac{\lambda_2}{\lambda_1}\right|\right)^k$$
    convergence linéaire
    On obtient $\lambda$ par le quotient de Rayleigl
    $$f_a(x_k) = \frac{x_k^*Ax_k}{x_k^*x_k} \rightarrow \lambda_1$$
    si $x_k \to v_1$ $f_A(x)$ est $C^1$ et $f(v_1) = \lambda_1$.

    Par l'expansion de Taylor
    $$\rho(v + \delta v) = \rho(v) + \bigoh(\|\delta v\|)$$
    la convergence de $\rho(x_k)$ vers $\lambda_1$ est aussi linéaire.
  \end{proof}
\end{mytheo}

\begin{myrem}
  \begin{itemize}
    \item $A$ pas $\diag$ $|\lambda_1| > |\lambda_2|$
    \item $x_0$ est ok si $a_1 \neq 0$
    \item Si $A = A^*$, $\rho(v + \delta v) = \rho(v) + \bigoh(\|\delta v\|^2)$
  \end{itemize}
\end{myrem}

% 14 nov

Rappel
\begin{mytheo}
  Si $A \in \Cnn$ est diagonalisable et $|\lambda_1| > |\lambda_2|$ alors pour presque tout $x_0$
  $$\lim_{k\to\infty} \angle(x_k,v_1) = 0.$$
\end{mytheo}

\begin{myalgo}
  A diagonalisable. On remplace $A$ par $(A-\mu I)^{-1} \in \Cnn$

  Les vecteurs propres de $(A - \mu I)^{-1}$ et $A$ sont les mêmes et les valeurs sont $\frac{1}{\lambda_i-\mu}$.
  \begin{proof}
    \begin{align*}
      Av_i & = \lambda_iv_i & i = 1,\ldots,n\\
      (A-\mu I)v_i & = (\lambda_i - \mu) v_i\\
      \frac{1}{\lambda_i - \mu} v_i & = (A - \mu I)^{-1}v_i
    \end{align*}
    $$\frac{1}{|\lambda_j - \mu|} > \frac{1}{|\lambda_k - \mu|}$$
  \end{proof}
  \begin{algorithm}
    \caption{INVIT ($x_0$ donné)}
    \label{algo:invit}
    \begin{algorithmic}
      \FOR{k = 0,1,\ldots}
      \STATE $(A - \mu I)z_{k+1} = x_k$
      \STATE $x_{k+1} = z_{k+1}/\|z_{k+1}\|$
      \ENDFOR
    \end{algorithmic}
  \end{algorithm}
  Ceci converge de façon linéaire vers $v_j$ corespondant à $\frac{1}{\lambda_j - \mu}$.
\end{myalgo}
\begin{mytheo}
  Soit $A \in \Cnn$ diagonalisable et $\frac{1}{\lambda_j - \mu}$ dominant.
  Alors pour presque tout $x_0$
  $$\lim_{k\to\infty} \angle(x_k,v_j) = 0$$
  et
  $$\angle(x_k,v_j) = \bigoh\left(\frac{|\lambda_j - \mu| < 1}{|\lambda_k - \mu|}\right)$$
\end{mytheo}
\begin{myrem}
  Quels $x_0$ ? $x_0 = \sum c_i v_i$ $c_j \neq 0$.
  \begin{itemize}
    \item Choix de $\mu$ est crucial.
    \item L'estimateur de $\lambda_j$ est $R(x_k) = x_k^*(A - \mu I)^{-1}x_k \approx \frac{1}{\lambda_j - \mu}$.
  \end{itemize}
\end{myrem}

\begin{myalgo}{Rayleigh quotient Iteration (RQI)}
  On restraint ici pour $A = A^*$. On utilise le shift $\mu_k = \frac{x_k^*Ax_k}{x_k^*x_k}$.
  \begin{algorithm}
    \caption{RQI ($x_0$ donné) et normalisé (pour faire comme le syllabus, on est pas obligé)}
    \label{algo:rqi}
    \begin{algorithmic}
      \FOR{k = 0,1,\ldots}
      \STATE $\left(A - \frac{x_k^*Ax_k}{x_k^*x_k} I\right)z_{k+1} = x_k$
      \STATE $x_{k+1} = z_{k+1}/\|z_{k+1}\|$
      \ENDFOR
    \end{algorithmic}
  \end{algorithm}
\end{myalgo}

\begin{mytheo}{RQI}
  Soit $A = A^*$. Alors l'itération RQI converge vers une direction propre pour presque tout $x_0$.
  La convergence ultime est dans ce cas \emph{cubique}.
\end{mytheo}

Comparer avec algorithme de Newton
$$F(x) = \left(A - \frac{x^*Ax}{x^*x} I\right) x = 0$$
aux vecteurs propres de $A$.

Newton appliqué
$$F(x) = \left(A - \frac{x^*Ax}{x^*x} I\right) x$$
Convergence requiert que $DF(s)$ non-singulier.
\emph{Problème}: $DF(s) = 0$ quand $F(s) = 0$.
\begin{proof}
  $F(\alpha s) = \alpha F(s)$.
  \begin{align*}
    F(s + \delta s) & = F(s) + \delta F(s)\\
                    & = F(s) + \delta DF(s) s + \bigoh(\delta^2).
  \end{align*}
\end{proof}

Une méthode modifiée projette la méthode de Newton sur un espace de dimension plus $n-1$.

\subsection{Itération de sous-espaces}
Idée $|\lambda_1| \geq |\lambda_2| \geq \ldots \geq |\lambda_p| > |\lambda_{p+1}| \geq \ldots $.
On va itérer avec $p$ vecteurs à la fois.

Soit $x = [x_1|\ldots|x_p] \in \C^{n \times p}$ alors
$$\mathcal{X} = \myim(X) = \col(X) = \myspan\{x_1,\ldots,x_p\}$$
$$A\mathcal{X} = \{y = Ax, x \in \mathcal{X}\} = A(\mathcal{X})$$
opérateur agissant sur un espace.

Si $\col(X) = \col(Y)$ alors
$$A \col(X) = A \col(Y)$$
\begin{proof}
  \begin{align*}
    \col(X) & = \col(Y) & \text{si } x = YM\det M\\
    A \col(X) & =\col(AX)\\
    A \col(Y) & = \col(AY) = \col(AXM) = \col(AX)
  \end{align*}
\end{proof}

La méthode la puissance générait
$$x_0,Ax_0,A^2x_0,\ldots$$
vecteurs propres dominants.
L'extension pour des bases de dimension $p$
$$\mathcal{X}_0,A\mathcal{X}_0,A^2\mathcal{X_0},\ldots$$
espace invariants dominants.

\begin{algorithm}
  \caption{$x_0$ initial (une base $\in \C^{n \times p}$ de $p$ vecteurs)}
  \label{algo:newton}
  \begin{algorithmic}
    \FOR{k = 0,1,\ldots}
    \STATE $z_{k+1} = Ax_k$
    \STATE $(x_{k+1},R_{k+1}) = QR(z_{k+1})$
    \ENDFOR
  \end{algorithmic}
\end{algorithm}

$$z_{k+1} = x_{k+1}R_{k+1}$$
donc
$$\col z_{k+1} = \col x_{k+1}$$
où $x_{k+1}$ est orthonormé.

$$Ax_k = x_{k+1} R_{k+1}$$
$\col(x_k+1) = A(\col(x_k))$.
Cet algorithme génère l'extension logique de la méthode de la puissance.

\begin{mytheo}
  Soit $A \in \Cnn$ diagonalisable et $|\lambda_p| > |\lambda_{p+1}|$ (val propres ordonnées) (gap entre $p$ et $p+1$).
  Pour tout $x_0$ tel que $\col(x_0) \cap \myspan\{v_{p+1},\ldots,v_n\} = \emptyset$ on a que
  $$\angle(A^k\mathcal{X}_0,\myspan\{v_1,\ldots,v_p\}) \leq c \left|\frac{\lambda_{p+1}}{\lambda_p}\right|^k$$
  convergence linéaire.
\end{mytheo}

\begin{myrem}
  $\angle(\mathcal{X},\mathcal{Y}) = \max_{x \in \mathcal{X},y\in \mathcal{Y}} \angle (x, y)$.
\end{myrem}

Si $Q = [\underbrace{Q_1}_{p}|\underbrace{Q_2}_{n-p}]\}n$ est unitaire $Q^*Q = I$ alors si $A\col(Q_1) \subseteq \col(Q_1)$ on a que
$$Q^* A Q =
\begin{pmatrix}
  A_{11} & A_{12}\\
  0 & A_{22}
\end{pmatrix}$$
$A_{11}$ a $\lambda_1,\ldots,\lambda_p$ et $A_{22}$ a $\lambda_{p+1},\ldots,\lambda_n$.

On peut réduire en deux sous-problèmes $A_{11}$ est $A_{22}$.

\subsection{Calculer toutes les valeurs propres (Francis (59))}

\begin{algorithm}
  \caption{QR}
  \label{algo:qr}
  \begin{algorithmic}
    \STATE $A := A$ (ou $Q_0^* A Q_0$)
    \FOR{k = 0,1,\ldots}
    \STATE $Q_{k+1}R_{k+1} = A_k$
    \STATE $A_{k+1} = R_{k+1}Q_{k+1}$
    \ENDFOR
  \end{algorithmic}
\end{algorithm}

Il découle que
$$A_{k+1} = Q_{k+1}^*A_kQ_{k+1}$$
similitude orthogonale (préserve les valeurs propres).
Si on définit
\begin{align*}
  \tilde{Q}_k & = (Q_0)Q_1Q_2\cdots Q_k\\
  \tilde{R}_k & = R_kR_{k-1}\cdots R_2R_1
\end{align*}
on a
\begin{align*}
  A_k & = \tilde{Q}_k^* A \tilde{Q}_k\\
  A\tilde{Q}_k & = \tilde{Q}_{k+1} \tilde{R}_{k+1} & \text{puisque } A_k = Q_{k+1}R_{k+1} (\text{induction})
\end{align*}

Il suit que pour tout $p$
\begin{align*}
  A\tilde{Q}_k
  \begin{pmatrix}
    I_p\\
    0
  \end{pmatrix}
 & = \tilde{Q}_{k+1}\tilde{R}_{k+1}
 \begin{pmatrix}
   I_p\\
   0
 \end{pmatrix}\\
 & = \tilde{Q}_{k+1}
 \begin{pmatrix}
   \tilde{R}_{11(k+1)}\\
   0
 \end{pmatrix}\\
 & = \tilde{Q}_{k+1}
 \begin{pmatrix}
   I_p\\
   0
 \end{pmatrix}
  \tilde{R}_{11(k+1)} & \tilde{R}_{11(k+1)} \text{ inversible}\\
 \col\left(A \tilde{Q}_k
   \begin{pmatrix}
     I_p\\
     0
   \end{pmatrix}
 \right) & =
   \col  \left(
     \tilde{Q}_{k+1}
     \begin{pmatrix}
       I_p\\
       0
     \end{pmatrix}
   \right)\\
   A\col\tilde{Q}_k
   \begin{pmatrix}
     I_p\\
     0
   \end{pmatrix} & =
     \col
     \tilde{Q}_{k+1}
     \begin{pmatrix}
       I_p\\
       0
     \end{pmatrix}
\end{align*}

Puisque ceci c'est l'itération des espaces sur les $p$ premières dolonnes de $\tilde{Q}_k$ elles convergeront vers un espace invariant
$$\tilde{Q}_k^*A\tilde{Q}_k = \text{matrice diagonale supérieure}.$$

\begin{algorithm}
  \caption{Accélération par shift}
  \label{algo:qrshift}
  \begin{algorithmic}
    \STATE $A = Q_0^*AQ_0$
    \FOR{k = 0,1,\ldots}
    \STATE $\sigma_k \in \C$
    \STATE $Q_{k+1}R_{k+1} = (A_k - \sigma_k I)$
    \STATE $A_{k+1} = R_{k+1}Q_{k+1} + \sigma_k I$
    \ENDFOR
  \end{algorithmic}
\end{algorithm}

Vérifier que $A_k = Q_{k+1}^* A_k Q_{k+1}$ mais la convergence dépendra maintenant de
$\left|\frac{\lambda_p-\sigma_k}{\lambda_{p+1}-\sigma_k}\right|$, convergera vers
$\begin{pmatrix}
  A_{11} & \vdots\\
  0 & \lambda_n
\end{pmatrix}$
et on recommence avec $A_{11}$.
On démontre que la convergence de $\lambda_n$ est quadratique si les $\sigma_k$ sont bien
choisis.

La transformation $Q_0$ peut être choisie telle que
$$A_0 = Q_0^*AQ_0 =
\begin{pmatrix}
  x & \cdots & \cdots & \cdots & x\\
  x & \ddots & \ddots & \ddots & \vdots\\
    & \ddots & \ddots & \ddots & \vdots\\
    &        & \ddots & x      & x\\
    &        &        & x      & x
\end{pmatrix}.$$
(forme de Heisenberg cfr. Arnoldi) $\bigoh(n^3)$ flops.

Cette forme est préservée par l'algorithme~\ref{algo:qrshift}.
On choisit $\sigma_k$ sur base de la matrice des 4 valeurs en bas à droite de $A_0$.

4.6 et suite pas à connaire pour l'examen mais à connaitre pour la culture générale.


\end{document}
